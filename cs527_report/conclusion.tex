% \section{Discussion}

% Many of the challenges were due to nitty-gritty engineering details. For example, actually getting the transformation right took a lot of time and debugging. Using thread-local and global variables for the CPU timers was also especially challenging. There were a few bugs that we had to work-around rather than fix, such as a race condition in the graphics code. This is the regular pain of working on a large research project.

% This also implies that many of the improvements we made were engineering details that we can upstream. Trying to actually use ILLIXR to do a course project has given myself (Sam speaking) more ideas for improving it than when I was thinking about improving it full-time!

\section{Conclusion}

In just one semester, we designed and implemented a system that answers \textbf{R.Q. 1} and \textbf{R.Q. 2}.
We learned from this system that existing video metrics do not transfer well to AR/VR.
We propose a novel video distance metric and our system provides a platform to test distance metrics on ILLIXR's trace.
We have set groundwork for future research to answer \textbf{R.Q. 3}.
